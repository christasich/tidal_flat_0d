%**************************************************************************
%*
%*  Paper: ``INSTRUCTIONS FOR AUTHORS OF LATEX DOCUMENTS''
%*
%*  Publication: 2018 Winter Simulation Conference Author Kit
%*
%*  Filename: writgd18.tex
%*
%*  Date: January 25, 2018   Time:  9:45 PM
%*      BASE of current version: Feb 01, 2010 (primary WSC'10 LaTeX file)
%*
%*  Word Processing System: TeXnicCenter and MiKTeX
%*
%*
%*  All files need the following
\input{wsc18style.tex}     % download from author kit.  Style files for wsc formatting. Don't remove this line - required for generating the final paper!

\documentclass{wscpaperproc}
\usepackage{latexsym}
%\usepackage{caption}
\usepackage{graphicx}
\usepackage{mathptmx}

%
%****************************************************************************
% AUTHOR: You may want to use some of these packages. (Optional)
\usepackage{amsmath}
\usepackage{amsfonts}
\usepackage{amssymb}
\usepackage{amsbsy}
\usepackage{amsthm}
%****************************************************************************



%
%****************************************************************************
% AUTHOR: If you do not wish to use hyperlinks, then just comment
% out the hyperref usepackage commands below.

%% This version of the command is used if you use pdflatex. In this case you
%% cannot use ps or eps files for graphics, but pdf, jpeg, png etc are fine.

\usepackage[pdftex,colorlinks=true,urlcolor=blue,citecolor=black,anchorcolor=black,linkcolor=black]{hyperref}

%% The next versions of the hyperref command are used if you adopt the
%% outdated latex-dvips-ps2pdf route in generating your pdf file. In
%% this case you can use ps or eps files for graphics, but not pdf, jpeg, png etc.
%% However, the final pdf file should embed all fonts required which means that you have to use file
%% formats which can embed fonts. Please note that the final PDF file will not be generated on your computer!
%% If you are using WinEdt or PCTeX, then use the following. If you are using
%% Y&Y TeX then replace "dvips" with "dvipsone"

%%\usepackage[dvips,colorlinks=true,urlcolor=blue,citecolor=black,%
%% anchorcolor=black,linkcolor=black]{hyperref}
%****************************************************************************



		



%
%****************************************************************************
%*
%* AUTHOR: YOUR CALL!  Document-specific macros can come here.
%*
%****************************************************************************
%****************************************************************************
%*
%*  Miscellaneous new command definitions I sometimes like to use...
%
\newcommand{\emdash}{\text{---}}
\newcommand{\ben}{\begin{enumerate}
\setlength{\parskip}{\smallskipamount}
\setlength{\parsep}{\smallskipamount}
\setlength{\itemsep}{\smallskipamount}
}
\newcommand{\een}{\end{enumerate}}
%\newcommand{\sect}[1]{\section*{{\large\bf#1}}} % Better ver. of \section*{}
%\renewcommand{\arraystretch}{1.5}
\newcommand{\comment}[1]{}
\newcount\hour \newcount\minutes \hour=\time \divide\hour by 60
\minutes=\hour \multiply\minutes by -60 \advance\minutes by \time
\def\Hour{\number\hour}
\def\Minutes{\ifnum\minutes<10 0\fi\number\minutes}
\def\HourMinutes{\Hour:\Minutes}
%*
%*   End of miscellaneous new command definitions
%*
%*****************************************************************************
% If you use theoremes
\newtheoremstyle{wsc}% hnamei
{3pt}% hSpace abovei
{3pt}% hSpace belowi
{}% hBody fonti
{}% hIndent amounti1
{\bf}% hTheorem head fontbf
{}% hPunctuation after theorem headi
{.5em}% hSpace after theorem headi2
{}% hTheorem head spec (can be left empty, meaning `normal')i

\theoremstyle{wsc}
\newtheorem{theorem}{Theorem}
\renewcommand{\thetheorem}{ \arabic{theorem}}
\newtheorem{corollary}[theorem]{Corollary}
\renewcommand{\thecorollary}{\arabic{corollary}}
\newtheorem{definition}{Definition}
\renewcommand{\thedefinition}{\arabic{definition}}


%#########################################################
%*
%*  The Document.
%*
\begin{document}

%***************************************************************************
% AUTHOR: AUTHOR NAMES GO HERE 
% FORMAT AUTHORS NAMES Like: Author1, Author2 and Author3 (last names)
%
%		You need to change the author listing below!
%               Please list ALL authors using last name only, separate by a comma except
%               for the last author, separate with "and"
%
\WSCpagesetup{Wilson}

% AUTHOR: Enter the title, all letters in upper case
\title{GUIDELINES ON WRITING A GOOD PAPER FOR THE \\
\textbf{\textit{PROCEEDINGS OF THE  WINTER SIMULATION CONFERENCE}}}

% AUTHOR: Enter the authors of the article, see end of the example document for further examples
\author{James R. Wilson \\ [12pt]
        North Carolina State University \\
        Edward P. Fitts Department of Industrial and Systems Engineering \\
        Raleigh, NC 27695-7906, USA \\
% Multiple authors are entered as follows.
% You may also need to adjust the titlevbox size in the preamble - search for titlevboxsize
}

\maketitle
% \input{wscpage2.tex}

\section*{ABSTRACT}
As an aid to authors who seek to improve the clarity and readability of
their papers in the \textit{Proceedings of the Winter Simulation
Conference}, this paper summarizes some useful guidelines on technical
writing, including current references on each topic that is discussed. 



\section{OUTLINE OF KEY CONSIDERATIONS}
\label{sec:intro}
Writing a clear, readable exposition of complex technical work is at least
as difficult as doing the work in the first place.  Given below is an
outline of key considerations to bear in mind during all stages of writing
a paper that will be reviewed for possible presentation at the Winter
Simulation Conference (WSC) as well as publication in the {\em Proceedings
of the Winter Simulation Conference}.  For questions about these
guidelines, please send e-mail to
\href{mailto:jwilson@ncsu.edu}{jwilson@ncsu.edu} or contact the proceedings
editors.
%\parindent=0em

\ben %I
\item[I.] Organizing the paper (what to do before beginning to write)
\ben %A

\item[A.] Analyze the situation---that is, the problem, the solution,
and the target audience.
\ben %1

\item[1.] Formulate the objectives of the paper.

\item[2.] Specify the scope of the paper's coverage of the subject and the
results to be discussed.  Orient the paper toward the theme of your session
as indicated either by the title of your session or by the instructions of
your session chair.  Also take into account the general focus of the track
containing your paper, which could be tutorials, case studies, vendors,
methodologies, domain-specific applications, or general applications.

\item[3.] Identify the target audience and determine the background knowledge
that you can assume for this particular group of people.  Introductory
tutorials are generally attended by newcomers who are interested in the
basics of simulation.  Advanced tutorials are designed to provide more
experienced professionals with a thorough discussion of special topics of
much current interest; and some special-focus sessions in this track are
designed to provide experts with an overview of recent fundamental advances
in simulation theory.  Methodology sessions are attended by professionals
who have at least an undergraduate-level background in computer simulation
techniques.  In the case studies and applications tracks, session attendees
are generally familiar with the area covered by their session.  Vendor
sessions may contain both new and experienced users of the relevant
software products. 

\newpage
\item[4.]
Formulate the most logical sequence for presenting the information
specified in item 2 to the readers identified in item 3.  For a discussion
of effective aids in organizing your paper (specifically, brainstorming,
clustering, issue trees, and outlining), see chapter 3 of
Matthews and Matthews (2014).  In structuring your presentation, keep the
following points in mind.

%\newpage
%\topmargin=-30.52pt
%\setlength{\topmargin}{-30.52pt}
\ben %a

\item[a.]
Introductory and advanced tutorials should have an educational perspective.
Within the advanced tutorials track, special-focus sessions should synthesize
the latest research results in a unified treatment of a given topic.

\item[b.]
Methodology contributions should provide state-of-the-art information on
proven techniques for designing, building, and analyzing simulation models.


\item[c.]
Application papers should relate directly to the practice of simulation, and
they should emphasize lessons of transferable value.

\een %a
\een %1

\item[B.]
Make outlines to organize your thoughts and then to plan both the
written and oral presentations of your work.  For excellent discussions of
the construction and use of various types of outlines, see the following:
chapter 1 of Menzel, Jones, and Boyd (1961); the sections titled ``Develop
an issue tree to assess presentation balance'' and ``Outline to develop the
paper's framework'' in chapter 3 of Matthews and Matthews (2014); and
chapter 3 of Pearsall and Cook (2010).
\ben %1

\item[1.] The introductory paragraph(s)
\ben %a

\item[a.] State the precise subject of the paper immediately.

\item[b.] State the problem to be solved.

\item[c.] Summarize briefly the main results and conclusions.

\item[d.] Tell the reader how the paper is organized.

\een %a

\item[2.] The main body of the paper
\ben %a

\item[a.] Include enough detail in the main body of the paper so that the
reader can understand what you did and how you did it; however, you should
avoid lengthy discussions of technical details that are not of general
interest to your audience.

\item[b.]  Include a brief section covering notation, background information,
and key assumptions if it is awkward to incorporate these items into the
introductory paragraph(s).

\item[c.] Include sections on theoretical and experimental methods as
required.  For an application paper, you should discuss the development of
the simulation model---including input data acquisition as well as design,
verification, validation, and actual use of the final simulation model.
For a methodological or theoretical paper that requires substantial
mathematical development, see Halmos (1970), Higham (1998), pages 1--8 of
Knuth, Larrabee, and Roberts (1989), Krantz (1997, 2001), or Swanson
(1999).  For standard mathematical notation used in engineering and the
sciences, see \textit{ISO 80000-2: Quantities and Units---Part 2:
Mathematical Signs and Symbols to Be Used in the Natural Sciences and
Technology} (ISO 2009) and Scheinerman (2011). 

\item[d.] Plan the results section to achieve the most effective mix
of text, figures, and tables in the presentation of the findings.  The
definitive reference on the design of tables and figures is Tufte (2001).

\een %a

\item[3.] The concluding paragraph(s)
\ben %a

\item[a.] 
Explain how the theoretical and experimental results relate to the original
problem.  State why these results are important.

\item[b.] 
Summarize any unresolved issues that should be the subject of future work.

\item[c.] 
State the final conclusions explicitly in plain language.

\een %a
\een %1
\een %A
%\newpage

\item[II.] Writing the paper
\ben %A

\item[A.]
Prepare an abstract that is concise, complete in itself, and
intelligible to a general reader in the field of simulation.  The
abstract may not exceed 150 words, and it should not contain any
references or mathematical symbols.
\ben %1

\item[1.] Summarize the objectives of the paper.

\item[2.] Summarize the results and conclusions.

\item[3.] State the basic principles underlying any new theoretical or
experimental methods that are developed in the paper.

\item[4.]
For complete instructions on the preparation of scientific abstracts, see
{\it Guidelines for Abstracts\/} (NISO 2010), pages 91--93 of Carter
(1987), page 5 of the {\it AIP Style Manual\/} (AIP 1990), or chapter 9 of
Gastel and Day (2016).
\een %1


\item[B.] Write the rest of the paper as though you were talking to a group of
interested colleagues about your work.
\ben %1

\item[1.] 
Strive for accuracy and clarity above all else.

\item[2.]
In writing the introduction, you should remember the following maxim:
\begin{quote}
The opening paragraph should be your best paragraph, and its opening
sentence should be your best sentence.  (Knuth, Larrabee, and Roberts 1989,
5)
\end{quote}
You cannot achieve such an ambitious goal on the first try; instead as you
add new sections to the paper, you should review and revise all sections
written so far.  For more on the spiral plan of writing, see pages 131--133
of Halmos (1970).
\ben
\item[a.]
Like the abstract, the introduction should be accessible to general
readers in the field of simulation.  
\item[b.]
For methodology papers and advanced tutorials,
substantially more advanced background may be assumed in the sections
following the introduction.
\een


\item[3.]
In constructing each sentence, place old and new information in the
respective positions where readers generally expect to find such
information.  For an excellent discussion of the principles of scientific
writing based on reader expectations, see Gopen and Swan (1990) and
Williams and Bizup (2014, 2017).
\ben %1

\item[a.] Place in the topic position (that is, at the beginning of the
sentence) the old information linking backward to the previous discussion.

\item[b.] Place in the stress position (that is, at the end of the sentence)
the new information you want to emphasize.

\item[c.] Place the subject of the sentence in the topic position, and follow
the subject with the verb as soon as possible.

\item[d.] Express the action of each sentence in its verb.
\een %1

\item[4.]
Make the paragraph the unit of composition.
\ben %a

\item[a.] Begin each paragraph with a sentence that summarizes the topic to be
discussed or with a sentence that helps the transition from the previous
paragraph.

\item[b.] Provide a context for the discussion before asking the reader to
consider new information.


\item[c.]
Avoid paragraphs of extreme length---that is, one-sentence paragraphs and
those exceeding 200 words.


\item[d.] Place the important conclusions in the stress position at the end of
the paragraph.
\een %a

\item[5.]
Allocate space to a topic in proportion to its relative importance.

\item[6.]
For methodology papers, emphasize the concepts of general applicability
that underlie the solution procedure rather than the technical details that
are specific to the problem at hand.  Supply only the technical details and
data that are essential to the development.

\item[7.]
For application papers, emphasize the new insights into the problem
that you gained from designing, building, and using the simulation model.

\item[8.]
Use standard technical terms correctly.
\ben %a
\item[a.]
For standard usage of mathematical terms, see James and James (1992) and
Borowski and Borwein (2002).  For example, a nonsquare matrix cannot be
called ``orthogonal'' even if any two distinct columns of that matrix are
orthogonal vectors.
\item[b.]
For standard usage of statistical terms, see Dodge (2003), Porkess (2005),
and Upton and Cook (2014).  For example, the probability density function
of a continuous random variable cannot be called a ``probability mass
function.'' 

\item[c.]
For standard usage of computer terms, see \textit{The Free On-Line
Dictionary of Computing} (Howe 1993) and \textit{Dictionary of Algorithms
and Data Structures} (Black 1998).

\item[d.]
For standard usage of industrial engineering terms, see {\it Industrial
Engineering Terminology\/} (IISE 2000).  For example, the time that a workpiece
spends in a manufacturing cell may be called ``cycle time'' or ``flow
time'' but not ``throughput time.''
\een %a

\item[9.]
Avoid illogical or potentially offensive sexist language.
See Miller and Swift (2001) for a commonsense approach to this issue.

\item[10.]
Strictly avoid the following---
\ben %a

\item[a.]
religious, ethnic, or political references;

\item[b.]
personal attacks;

\item[c.]
excessive claims about the value or general applicability of your work; and

\item[d.]
pointed criticism of the work of other people.
\een %a

Such language has no place in scientific discourse under any circumstances,
and it will not be tolerated by the proceedings editors.  With
respect to vendor sessions, items c and d immediately above require authors
to avoid invidious comparisons of their products with competing products.


\item[11.]
In writing the final section of the paper containing conclusions and
recommendations for future work, you should keep in mind the following
maxim:
\begin{quote}
The mark of a good summary is revelation: ``Remember this, reader?  And
that?  Well, here's how they fit together.'' (van Leunen 1992, 116)  
\end{quote}

\een %1



\item[C.] For each table, compose a caption that briefly summarizes the
content of the table.  Comment explicitly in the text on the significance
of the numbers in the table; do not force the reader to guess at your
conclusions.  See sections 3.46--3.85 of {\em The Chicago Manual of Style\/}
(University of Chicago Press 2010) or chapter 16 of Gastel and Day (2016)
for a comprehensive discussion of how to handle tables.

\item[D.]For each figure, compose a caption (or legend) that explains every
detail in the figure---every curve, point, and symbol.  See the {\em AIP
Style Manual\/} (AIP 1990) or chapters 17 and 18
of Gastel and Day (2016) for excellent examples.

\item[E.] Revise and rewrite until the truth and clarity of every sentence are
unquestionable.

\ben %1
\item[1.]
For questions about the rules of English grammar and usage, see Bernstein
(1965), Butterfield (2015), Fowler and Aaron (2016), Garner (2016), Hale
(2013), O'Conner (2009), Strunk and White (2000), the \textit{Oxford
English Dictionary} (Simpson and Weiner 1989), and {\it Webster's Third New
International Dictionary of the English Language, Un\-a\-bridged\/} (Gove 1993).
\item[2.]
For those who use English as a second language, particularly helpful
references are Booth (1993), Fowler and Aaron (2016), Huckin and Olsen
(1991), and Yang (1995).

\item[3.]
For guidelines on how to edit your own writing effectively, see Cook (1985).

\item[4.]
For a comprehensive discussion of all aspects of scientific writing, see
Alley (1996) and Gastel and Day (2016).
\een %1

\item[F.]
Prepare a complete and accurate set of references that gives adequate credit
to the prior work upon which your paper is based.
\ben %1
\item[1.]
The author-date system of documentation is required for all papers
appearing in the {\it Proceedings of the Winter Simulation Conference}.
Chapter 15 of {\it The Chicago Manual of Style\/} (University of Chicago Press 2010) provides
comprehensive, up-to-date information on this citation system.

\item[2.]
In preparing your list of references, you should strive for completeness,
accuracy, and consistency.  Using the information provided in your list of
references, the interested reader should be able to locate each source of
information cited in your paper.

\item[3.]
For complete instructions on citing electronic sources, see sections
14.4--14.13 of \textit{The Chicago Manual of Style} (University of Chicago Press 2010).  For example, sections 14.5 and 14.6 contain basic
information on uniform resource locators (URLs) and Digital Object
Identifiers (DOIs), respectively; and section 14.12 provides useful rules
for breaking a URL or a DOI across two or more lines either in the text or
in the list of references.  Many specific examples of citations for various
types of electronic sources can be found throughout chapters 14 and 15 of
\textit{The Chicago Manual of Style} (University of Chicago Press 2010).

\item[4.] The final electronic version of your paper---that is, the portable
document format (PDF) file ultimately produced from the Word or \LaTeX\
source file of your paper---may include external hyperlinks referring to
some of the electronic sources cited in the paper that are accessible
online.
\ben
\item[a.]
If an external hyperlink is live, then it is colored blue; and when viewing
the PDF file of your paper on a computer, the reader may select (click)
that hyperlink for immediate online access to the cited material.  More
specifically, selecting (clicking) a live external hyperlink will activate
the reader's web browser so that, if all goes well, the cited source of
information will be displayed in the web browser.  A live external hyperlink
may also be used to activate the reader's e-mail software for sending a
message to a specific e-mail address; for example, see the hyperlink given
in the first paragraph of this document.
\item[b.]
If an external hyperlink is not live, then it is colored black; and such a
hyperlink merely displays the URL or DOI of the cited material without
providing a mechanism for immediate online access to that material.  
\item[c.]
If you use external hyperlinks in your paper, then you must ensure that the
text displayed for each external hyperlink is correct and complete so that
a reader who has only a hard copy of the paper can still access the cited
material by (carefully) typing the relevant displayed text of the hyperlink
into the address bar of a web browser or e-mail program.  Remember that
your responsibility for the accuracy and completeness of each hyperlink in
your paper parallels your responsibility for the accuracy and completeness
of each conventional citation of a nonelectronic source---neither the
editors nor the publisher of the proceedings can verify any of this
information for you. 
\een

\een%1


\item[G.] 
See Wilson (2002) for a discussion of the following ethical and
``strategic'' considerations in writing a scientific paper that will be
considered for publication in a peer-reviewed journal or conference
proceedings such as the \textit{Proceedings of the Winter Simulation
Conference}:
\ben%1
\item[1.]
achieving a consensus among collaborators on who should be a coauthor of
the paper;
\item[2.]
achieving a consensus among coauthors on the order of authorship in the paper's
byline; and
\item[3.]
writing the paper so as to anticipate and answer key questions that will
be asked by the paper's referees and readers.
\een%1
\een %A

\item[III.] Achieving a natural and effective style
\ben %A

\item[A.] Alfred North Whitehead memorably expressed the gist of
the matter of writing style:    
\begin{quote}
Finally, there should grow the most austere of all mental qualities; I
mean the sense for style.  It is an aesthetic sense, based on admiration
for the direct attainment of a foreseen end, simply and without waste.
Style in art, style in literature, style in science, style in logic,
style in practical execution have fundamentally the same aesthetic
qualities, namely attainment and restraint.  The love of a subject in
itself and for itself, where it is not the sleepy pleasure of pacing a
mental quarter-deck, is the love of style as manifested in that study. 
\smallskip\parindent=1.2em

Here we are brought back to the position from which we started, the
utility of education.  Style, in its finest sense, is the last acquirement
of the educated mind; it is also the most useful. It pervades the whole
being.  The administrator with a sense for style hates waste; the engineer
with a sense for style economises his material; the artisan with a sense
for style prefers good work.  Style is the ultimate morality of mind.
(Whitehead 1929, 12)
\end{quote}

Kurt Vonnegut made the following equally trenchant observation on
writing style. 
\begin{quote}
Find a subject you care about and which you in your heart feel others
should care about. It is this genuine caring, and not your games with
language, which will be the most compelling and seductive element in your
style.  (Vonnegut 1985, 34)
\end{quote}


Strunk and White (2000), Williams and Bizup (2014, 2017), and Zinsser
(2006) are excellent references on achieving a natural and effective
writing style.


\item[B.] Contrast the following descriptions of an experiment in optics:
\ben %1

\item[1.] I procured a triangular glass prism, to try therewith the celebrated
phenomena of colors.  And for that purpose, having darkened my laboratory, and
made a small hole in my window shade, to let in a convenient quantity of the
sun's light, I placed my prism at the entrance, that the light might be
thereby refracted to the opposite wall.  It was at first a very pleasing
diversion to view the vivid and intense colors produced thereby.

\item[2.]
For the purpose of investigating the celebrated phenomena of
chromatic refrangibility, a triangular glass prism was procured.
After darkening the laboratory and making a small aperture in an
otherwise opaque window covering in order to ensure that the optimum
quantity of visible electromagnetic radiation (VER) would be admitted
from solar sources, the prism was placed in front of the aperture for
the purpose of reflecting the VER to the wall on the opposite side of
the room.  It was found initially that due to the vivid and intense
colors which were produced by this experimental apparatus, the overall
effect was aesthetically satisfactory when viewed by the eye.

\een %1

The most striking difference between these two accounts of the experiment
is the impersonal tone of the second version.  According to version 2,
literally nobody performed the experiment.  Attempting to avoid the first
person, the author of version 2 adopted the third person; this in turn
forced the author to use passive verbs. As Menzel, Jones, and Boyd (1961,
79) point out, ``Passive verbs increase the probability of mistakes in
grammar; they start long trains of prepositional phrases; they foster
circumlocution; and they encourage vagueness.''  Notice the dangling
constructions in the second sentence of version 2.  Version 1 was written
by Isaac Newton (1672, 3076).  Even though it was written over 340 years
ago, Newton's prose is remarkable for its clarity and readability.

\item[C.] To achieve a natural and effective writing style, you should adhere
to the following principles that are elaborated in chapter 5 of Menzel, Jones,
and Boyd (1961):
\ben %1

\item[1.] Write simply.

\item[2.] Use the active voice.

\item[3.]
Use plain English words rather than nonstandard technical jargon or foreign
phrases.

\item[4.]
Use standard technical terms correctly.

\item[5.]
Avoid long sentences and extremely long (or short) paragraphs.

\item[6.]
Avoid slavish adherence to any set of rules for technical writing, including
the rules enumerated here.

\item[7.]
Remember that the main objective is to communicate your ideas clearly to your
audience.

\een %1
\een %A
\een %I

\section{SUMMARY}

In writing a paper for publication in the \textit{Proceedings of the Winter
Simulation Conference}, the author should keep in mind the key
considerations outlined in this paper.  Questions and suggestions for
improvement of this document are welcome.

\section*{ACKNOWLEDGMENTS}

These guidelines are based on a similar document prepared by James O.\
Henriksen, Stephen D.\ Roberts, and James R.\ Wilson for the {\em Proceedings
of the 1986 Winter Simulation Conference}.


\section*{REFERENCES}
\begin{hangref}

\item
AIP (American Institute of Physics).  1990.  {\em AIP Style Manual}. 4th
ed.  New York: AIP\@.  \href{http://kmh-lanl.hansonhub.com/AIP_Style_4thed.pdf}%
{\url{http://kmh-lanl.hansonhub.com/AIP_Style_4thed.pdf}}\textcolor{black}{.}

\item
Alley, M\@.  1996.  \textit{The Craft of Scientific Writing}.  3rd ed.  New
York: Springer.

\item
Bernstein, T. M\@.  1965.  {\it The Careful Writer: A Modern Guide to English
Usage}.  New York: Atheneum.

\item
Black, P. E., ed\@.  1998.  \textit{Dictionary of Algorithms and Data
Structures}. Washington, DC: National Institute of Standards and
Technology.   \href{http://www.nist.gov/dads/}%
{\url{http://www.nist.gov/dads}}\textcolor{black}{.}

\item
Booth, V\@. 1993.  {\it Communicating in Science: Writing a Scientific Paper
and Speaking at Scientific Meetings}.  2nd ed.  Cambridge: Cambridge University
Press.

\item
Borowski, E. J., and J. M. Borwein.  2002.  \textit{Collins Web-linked
Dictionary of Mathematics}.  2nd ed.  Glasgow: HarperCollins.

\item
Butterfield, J.  2015.  {\em Fowler's Dictionary of Modern English Usage}.
4th ed.\ New York: Oxford University Press.


\item
Carter, S. P\@. 1987. {\it Writing for Your Peers: The Primary Journal Paper}.
New York: Praeger.

\item
Cook, C. K\@.  1985.  {\it Line by Line: How to Improve Your Own Writing}.
Boston: Houghton Mifflin Co.

\comment{
\item
Day, R. A., and N. Sakaduski. 2011.  \textit{Scientific English: A Guide
for Scientists and Other Professionals}.  3rd ed.  Santa Barbara, CA: Greenwood.
}

\item
Dodge, Y\@. ed.  2003.  \textit{The Oxford Dictionary of Statistical
Terms}.  6th ed.  New York: Oxford University Press.
  

\item
Fowler, H. R., and J. E. Aaron.  2016.  {\em The Little, Brown Handbook}.
13th ed.  Boston: Pearson.

\comment{
\item
Fowler, H. W\@.  1965.  \textit{A Dictionary of Modern English Usage}.
2nd ed.  Revised and edited by Sir Ernest Gowers. New York: Oxford University Press.
}

\item
Garner, B. A\@.  2016. \textit{The Chicago Guide to Grammar, Usage, and
Punctuation}.  Chicago: University of Chicago Press.

\item
Gastel, B., and R. A. Day. 2016.  {\it How to Write and Publish a
Scientific Paper}.  8th ed.  Santa Barbara, CA: Greenwood.


\item
Gopen, G. D., and J. A. Swan.  1990.  ``The Science of Scientific Writing.''  {\em
American Scientist\/} 78(6):550--558.

\item
Gove, P. B\@.  1993.  {\em Webster's Third New International Dictionary of the English Language,
Unabridged}.  Springfield, MA: Merriam-Webster.  
\href{http://unabridged.merriam-webster.com}{%
\url{http://unabridged.merriam-webster.com}}\textcolor{black}{.}


\item
Hale, C\@.  2013.  \textit{Sin and Syntax: How to Craft Wickedly
Effective Prose}.  New York: Three Rivers Press.

\item
Halmos, P. R\@.  1970. ``How to Write Mathematics.''
\textit{L'Enseignement Math\'ematique} 16(2):123--152. 
\href{http://www4.ncsu.edu/~jwilson/files/halmos70em.pdf}%
{\url{http://www4.ncsu.edu/~jwilson/files/halmos70em.pdf}}\textcolor{black}{.}


\item
Higham, N. J\@.  1998.  {\it Handbook of Writing for the Mathematical
Sciences}.  2nd ed.  Philadelphia: Society for Industrial and Applied
Mathematics.

\item
Howe, D., ed\@.  1993.  \textit{The Free On-Line Dictionary of Computing}.
London: Imperial College Department of Computing.    
\href{http://foldoc.org/}%
{\url{http://foldoc.org}}\textcolor{black}{.}


\item
Huckin, T. N., and L. A. Olsen.  1991.  {\it Technical Writing and
Professional Communication for Nonnative Speakers of English}.  2nd ed. New
York: McGraw-Hill.


\item
IISE (Institute of Industrial and Systems Engineers).  2000. {\it
Industrial Engineering Terminology}.  Rev.\ ed.  Norcross, GA: Institute of
Industrial Engineers.
\href{http://www.iise.org/details.aspx?id=645}%
{\url{http://www.iise.org/details.aspx?id=645}}\textcolor{black}{.}


\item
ISO (International Organization for Standardization).  2009.  \textit{ISO
80000-2: Quantities and Units: Part 2: Mathematical Signs and Symbols to
Be Used in the Natural Sciences and Technology}.  Geneva: International
Organization for Standardization.
\href{http://www4.ncsu.edu/~jwilson/files/mathsigns.pdf}%
{\url{http://www4.ncsu.edu/~jwilson/files/mathsigns.pdf}}\textcolor{black}{.}

 
\item
James, R. C., and G. James.  1992.  {\it Mathematics Dictionary}. 5th ed.
New York: Chapman \& Hall.

\item
Knuth, D. E., T. Larrabee, and P. M. Roberts.  1989.  \textit{Mathematical
Writing}.  Washington, DC: Mathematical Association of America.  
\href{http://www4.ncsu.edu/~jwilson/files/mathwriting.pdf}%
{\url{http://www4.ncsu.edu/~jwilson/files/mathwriting.pdf}}\textcolor{black}{.}

\item
Krantz, S. G\@.  1997.  {\it A Primer of Mathematical Writing: Being a
Disquisition on Having Your Ideas Recorded, Typeset, Published, Read, and
Appreciated}.  Providence, RI: American Mathematical Society.

\item
Krantz, S. G\@.  2001.  \textit{Handbook of Typography for the
Mathematical Sciences}.  Boca Raton, FL: Chapman and Hall/CRC.

\item
Matthews, J. R., and R. W. Matthews.  2014.  {\it
Successful Scientific Writing: A Step-by-Step Guide for the Biological
and Medical Sciences}.  4th ed.  Cambridge: Cambridge University Press.


\item
Menzel, D. H., H. M. Jones, and L. G. Boyd.  1961.  {\em Writing
a Technical Paper}.  New York: McGraw-Hill Book Company.


\item
Miller, C., and K. Swift.  2001.  {\em The Handbook of Nonsexist Writing}.
2nd ed.  San Jose, CA: iUniverse.com.



\item
Newton, I\@.  1672.  ``New Theory of Light and Colors.''  {\em Philosophical
Transactions} 6(80):3075--3087.


\item
NISO (National Information Standards Organization).
  2010.  {\it ANSI/NISO Z39.14-1997 (R2009): Guidelines for Abstracts}. Bethesda, MD: NISO Press.  
\href{http://www.oregon.gov/ODOT/Programs/ResearchDocuments/Abstract-Guidelines.pdf}{%
\url{http://www.oregon.gov/ODOT/Programs/ResearchDocuments/Abstract-Guidelines.pdf}}\textcolor{black}{.}

\item
O'Conner, P. T\@.  2009.  \textit{Woe Is I: The Grammarphobe's Guide to
Better English in Plain English}.  3rd ed.  New York: Riverhead Books.

\item
Pearsall, T. E., and K. C. Cook.  2010.  \textit{The Elements of Technical
Writing}.  3rd ed.  New York: Longman.

\item
Porkess, R., ed.  2005.  \textit{Collins Web-linked Dictionary of
Statistics}.  2nd ed.  Glasgow: HarperCollins.

\item
Scheinerman, E. R\@.  2011.  \textit{Mathematical Notation: A Guide for
Engineers and Scientists}.  Seattle: CreateSpace.
\href{https://www.ams.jhu.edu/ers/books/mathematical-notation}%
{\url{https://www.ams.jhu.edu/ers/books/mathematical-notation}}\textcolor{black}{.}

\item
Simpson, J.~A., and E.~S.~C. Weiner, eds. 1989.
{\em Oxford English Dictionary}. 2nd ed.
Oxford, UK: Oxford University Press.
\href{http://www.oed.com}{%
\url{http://www.oed.com}}\textcolor{black}{.}


\item
Strunk, W., Jr., and E. B. White.  2000.  {\em The Elements of Style}.
4th ed.  Boston: Allyn and Bacon.


\item
Swanson, E\@.  1999.  {\it Mathematics into Type}.  Updated edition by
A. O'Sean and A. Schleyer.  Providence, RI: American
Mathematical Society.

\item
University of Chicago Press. 2010.
\textit{The Chicago Manual of Style}.  16th ed.  Chicago: University of Chicago Press.
\href{http://www.chicagomanualofstyle.org}{%
\url{http://www.chicagomanualofstyle.org}}\textcolor{black}{.}




\item
Tufte, E. R\@.  2001.  {\em The Visual Display of Quantitative Information}.
2nd ed.  Cheshire, CT: Graphics Press.


\item
Upton, G., and I. Cook.  2014.  \textit{A Dictionary of
Statistics}.  3rd ed.  Oxford: Oxford University Press.


\item
van Leunen, M.-C\@.  1992.  \textit{A Handbook for Scholars}.  Rev.\ ed.
New York: Oxford University Press.


\item
Vonnegut, K\@.  1985.  ``How to Write with Style.''  In \textit{How to Use
the Power of the Printed Word}, edited by B. S. Fuess, Jr., 33--38.
Garden City, NY: Anchor Press/Doubleday.  
\href{http://www.novelr.com/2008/08/16/vonnegut-how-to-write-with-style}{%
\url{http://www.novelr.com/2008/08/16/vonnegut-how-to-write-with-style}}\textcolor{black}{.}





\item
Whitehead, A. N\@.  1929.  ``The Aims of Education.''  In \textit{The Aims of
Education and Other Essays}, 12.  New York: The Free Press.

\item
Williams, J. M., and J. Bizup.  2014.  \textit{Style: The Basics of Clarity and
Grace}.  5th ed.  Boston: Longman. 


\item
Williams, J. M., and J. Bizup.  2017.  {\it Style: Lessons in Clarity and Grace}.
12th ed.  Boston: Pearson.

\item%[(68)]
Wilson, J. R\@.  2002.  ``Responsible Authorship and Peer Review.''
\textit{Science and Engineering Ethics} 8(2):155--174.  
\href{http://www4.ncsu.edu/~jwilson/files/wilson02see.pdf}%
{\url{http://www4.ncsu.edu/~jwilson/files/wilson02see.pdf}}\textcolor{black}{.}


\item
Yang, J. T\@.  1995.  \textit{An Outline of Scientific Writing: For
Researchers with English as a Foreign Language}.  Singapore: World
Scientific.  

\item
Zinsser, W\@.  2006.  {\it On Writing Well: The Classic Guide to Writing
Nonfiction}.  7th ed.  New York: HarperCollins Publishers.
\end{hangref}


\section*{AUTHOR BIOGRAPHY}
 
\noindent {\bf JAMES R. WILSON} 
is a professor in the Edward P. Fitts Department of Industrial and Systems
Engineering at North Carolina State University.  His current research
interests are focused on probabilistic and statistical issues in the design
and analysis of simulation experiments.  He has held the following
editorial positions: departmental editor of \textit{Management Science}
(1988--1996); area editor of \textit{ACM Transactions on Modeling and
Computer Simulation} (1997--2002); guest editor of a special issue of
\textit{IIE Transactions} honoring Alan Pritsker (1999--2001); and
Editor-in-Chief of \textit{ACM Transactions on Modeling and Computer
Simulation} (2004--2010).  He served The Institute of Management Sciences
College on Simulation (now the INFORMS Simulation Society) as
secretary-treasurer (1984--1986), vice president (1986--1988), and
president (1988--1990).  His activities in the Winter Simulation Conference
(WSC) include service as proceedings editor
(1986), associate program chair (1991), and program chair (1992).  During
the period 1997--2004, he was a member of the WSC Board of Directors
corepresenting the INFORMS Simulation Society; and he served as secretary
(2001), vice chair (2002), and chair (2003).  During the period 2006--2009,
he was a trustee of the WSC Foundation, serving as secretary (2006),
vice-president (2007), and president (2008).  He is a member of ACM, ASA,
ASEE and SCS; and he is a fellow of IIE and INFORMS\@.  His e-mail address
is
\href{mailto:jwilson@ncsu.edu}{jwilson@ncsu.edu}\textcolor{black}{,} and his web address is 
\href{http://www.ise.ncsu.edu/jwilson}{http://www.ise.ncsu.edu/jwilson}\textcolor{black}{.}



\end{document}

