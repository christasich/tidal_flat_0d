\documentclass[9pt,twocolumn,twoside,lineno]{pnas-new}
% Use the lineno option to display guide line numbers if required.

% Preamble
\usepackage{siunitx}

% Document parameters
\DeclareSIUnit\year{yr}
\DeclareSIUnit\foot{ft}

\templatetype{pnasresearcharticle} % Choose template
% {pnasresearcharticle} = Template for a two-column research article
% {pnasmathematics} %= Template for a one-column mathematics article
% {pnasinvited} %= Template for a PNAS invited submission

\title{Sediment model manuscript}

% Use letters for affiliations, numbers to show equal authorship (if applicable) and to indicate the corresponding author
\author[a,1]{Christopher M. Tasich}
\author[a,2]{Jonathan Gilligan}
\author[a,3]{Steven L. Goodbred}

\affil[a]{Vanderbilt University}

% Please give the surname of the lead author for the running footer
\leadauthor{Tasich}

% Please add a significance statement to explain the relevance of your work
\significancestatement{Authors must submit a 120-word maximum statement about the significance of their research paper written at a level understandable to an undergraduate educated scientist outside their field of speciality. The primary goal of the significance statement is to explain the relevance of the work in broad context to a broad readership. The significance statement appears in the paper itself and is required for all research papers.}

% Please include corresponding author, author contribution and author declaration information
\authorcontributions{Author contributions: C.M.T., J.G., and S.L.G. designed research;  C.M.T. and J.G. performed research; C.M.T. analyzed data; C.M.T. wrote the paper; and C.M.T., J.G., and S.L.G. provided discussion and input on paper}
\authordeclaration{The authors declare no conflict of interest.}
\correspondingauthor{\textsuperscript{1}To whom correspondence should be addressed. E-mail: chris.tasich\@vanderbilt.edu}

% At least three keywords are required at submission. Please provide three to five keywords, separated by the pipe symbol.
\keywords{aggradation $|$ sea level rise $|$ tidal river management $|$ polders $|$ tidal amplification} 

\begin{abstract}
Please provide an abstract of no more than 250 words in a single paragraph. Abstracts should explain to the general reader the major contributions of the article. References in the abstract must be cited in full within the abstract itself and cited in the text.
\end{abstract}

\dates{This manuscript was compiled on \today}
\doi{\url{www.pnas.org/cgi/doi/10.1073/pnas.XXXXXXXXXX}}

\begin{document}

\maketitle
\thispagestyle{firststyle}
\ifthenelse{\boolean{shortarticle}}{\ifthenelse{\boolean{singlecolumn}}{\abscontentformatted}{\abscontent}}{}

% If your first paragraph (i.e. with the \dropcap) contains a list environment (quote, quotation, theorem, definition, enumerate, itemize...), the line after the list may have some extra indentation. If this is the case, add \parshape=0 to the end of the list environment.
\dropcap{S}ea level rise threatens the densely populated and ecologically significant low-lying, coastal region of Bengal. Mitigating the effects of sea level rise (SLR) in this region will be especially difficult considering current widespread land management practices and the ongoing geopolitical tension between India and Bangladesh.

Coastal Bengal is situated within the delta formed by the confluence of the Ganges and Brahmaputra rivers and straddles the border between West Bengal, India to the east and Bangladesh to the west. The region is home to \textasciitilde30 million people\cite{centerforinternationalearthscienceinformationnetwork-ciesin-columbiauniversityPopulationEstimationService2018} and the ecologically critical Sundarbans mangrove forest. Since the 1960s, the region has seen a vast transformation through the reduction in natural mangrove habitat and the widspread construction of earthen embankments. These embankments surround large swaths of clearcut land, known as polders, that accomodate the swelling population of the region.

While, preventing regular inundation by spring high tides, the creation of these embankments have had the unintended consequence of starving the interior of the polders of fresh sediment. Without this sediment, polder interiors have compacted resulting in signficant elevation offset (\SIrange{1.0}{1.5}{\meter}) relative to the natural mangrove forest\cite{auerbachFloodRiskNatural2015}. Polder elevations often sit precariously below local mean high water (MHW) levels leading to persistent waterlogging. Furthermore, many of these embankments are in disrepair and susceptible to breaching especially by storm surge as was the case with Cyclone Sidr (2007), Cyclone Aila (2009), and recently with Cyclone Amphan (2020).

Tidal river management (TRM) has been proposed as a possible augmentation of current land management practices to alleviate some of the issues caused by poldering. Under TRM, embankments are seasonally breached to allow tidal water inundation and sediment aggradation. Many low-lying areas may recover to an acceptable elevation within 5 to 10 years, though, some may take longer. There are a host of local socioeconomic, political, and governance consideration that may influence the success of TRM. Here, we neglect those considerations and focus on the general feasibility of TRM in regards to SLR and the sediment supply of the GB system. Future studies will focus on the social dynamics surround TRM.

Coastal Bengal is often seen as one of the most at-risk regions for SLR due to climate change. Infographics often depict large swaths of the Bengal coastline flooded under different SLR scenarios. However, this overly simplifies the threats to the region and neglects the significant sediment contribution of the GB system in mainintaining the natural elevation.

Estimates for increases in Relative Mean Sea Level (RMSL) in the GB delta range from \SIrange{2.8}{8.8}{\milli\meter\per\year}. However, RMSL neglects the widening of the tidal range in the polder region. On average, local high water levels in the polder region are increasing at a rate of \SI{15.9}{\milli\meter\per\year}\cite{pethickRapidRiseEffective2013}. While SLR is of paramount importance, tidal range amplification is the more imminent threat to the region. This is especially important considering the Bangladeshi government's recent reinvestment in poldering with a \$400 million loan from the World Bank for the Coastal Embankment Improvement Project - Phase I (CEIP-I).

As for the natural mangrove system, it is unclear how changing water levels will affect elevation. Some studies have shown that the region is incredibly resilient to increasing water levels due, in large part, to the abudant sediment supply of the GB system. This sediment is delivered to the platform periodically during spring high tide which helps maintain an equilibirum elevation approximately equivalent to mean higher high water (MHHW). But, this large volume of sediment delivered to coastal Bengal is not guaranteed.

Water has long been the focus of the geopolitical disputes between India and Bangladesh. However, the reduction in waterflow across the border portends a significant decrease in sediment flux. Estimates suggest sediment flux may be reduced by \SIrange{39}{75}{\percent} for the Ganges and \SIrange{9}{25}{\percent} for the Brahmaputra resulting in a change in aggradation from \SI{3.6}{\milli\meter\per\year} to \SI{2.5}{\milli\meter\per\year}\cite{higginsRiverLinkingIndia2018}.

The combination of increasing water levels and decreasing sediment supply may further intensify an already dire situation. Here, we use a zero-dimensional mass balance model of sediment aggradation to understand the impact that increasing water levels and decreasing sediment flux will have on the regions equilibrium elevation and consequently its resilience to climate change. We consider both the resilience of the natural mangrove system and the ability of the polder system to recover to a more resilient elevation through TRM.

% \section*{Model design}

\subsection*{Numerical model}

We modeled tidal platform elevation ($\zeta$) using a zero-dimensional mass balance model using the basic formulation provided by Krone \cite{kroneMethodSimulatingMarsh1987} and further refined by Allen \cite{allenSaltmarshGrowthStratification1990}, French \cite{frenchNumericalSimulationVertical1993}, and Temmerman et al. \cite{temmermanModellingLongtermTidal2003,temmermanModellingEstuarineVariations2004}. The rate of tidal platform elevation change is described as
\begin{equation}\label{mass_bal_eq}
	\frac{d \eta}{d t} = \frac{d S_M}{d t} + \frac{d S_O}{d t} + \frac{d P}{d t} + \frac{d M}{d t},
\end{equation}
where $S_M$ is mineral sedimentation, $S_O$ is organic matter sedimentation, $P$ is compaction, and $M$ is tectonic subsidence. Each term of equation \ref{mass_bal_eq} can be further expanded.

We approximate $S_M$ as
\begin{equation}\label{min_sed_eq}
	S_M(t) = \int{\frac{w_{s}C(t)}{\rho_b}dt},
\end{equation}
where $w_s$ is the characteristic settling velocity of a given grain size, $C(t)$ is the depth-averaged and time-varying sediment concentration in the water column, and $\rho_b$ is the dry bulk density of the sediment. We assume there is no resuspension of mineral sediment which is consistent with Krone's \cite{kroneMethodSimulatingMarsh1987} initial formulation.

$w_s$ is calculated for a given grain size by using Stokes' law to determine the terminal velocity of a sphere falling through a fluid given by
\begin{equation}\label{ws_eq}
	w_s =  \frac{2}{9}\frac{\rho_p - \rho_f}{\mu}gR^2
\end{equation}

where $\rho_p$ is the mass density of the particle or grain, $\rho_f$ is the mass density of the fluid, $\mu$ is the dynamic viscosity of the fluid, $g$ is the acceleration due to gravity, and $R$ is the radius of the grain. This is approximation for $w_s$ is consistent with previous similar studies \cite{allenSaltmarshGrowthStratification1990,temmermanModellingLongtermTidal2003,temmermanModellingEstuarineVariations2004}. We assume basic properties of water for $\rho_f$ (\SI{1000}{\kilo\gram\per\cubic\meter}) and $\mu$ (\SI{1e-3}{\kilo\gram\per\meter\per\second}) for simplicity. Salinity does vary seasonally which will change these values, but had little affect on the model output.

We capture the temporal variation of $C(t)$ through the mass balance given as
\begin{equation}\label{C_eq}
	\frac{d[h(t)-\zeta(t)]C(t)}{dt} = -w_sC(t)+C_{in}\frac{dh}{dt},
\end{equation}
where $h$ is the height of the water column and $C_{in}$ is the incoming suspended sediment concentration of the adjacent water column. We consider $\zeta$ to be a function of time and update it at every timestep which differs from previous studies \cite{kroneMethodSimulatingMarsh1987,allenSaltmarshGrowthStratification1990,frenchNumericalSimulationVertical1993,temmermanModellingLongtermTidal2003,temmermanModellingEstuarineVariations2004,frenchTidalMarshSedimentation2006} which only update $\zeta$ after every tidal cycle. The physical interpretation of equation \ref{C_eq} is that the first term is the mass flux above an area on the tidal platform, the second term is the mass flux extracted from the water column, and the third term is the mass flux from the adjacent water column. Further derivation of equation \ref{C_eq} results in

\begin{equation}\label{C_eq2}
	\frac{dC}{dt}[h(t) - \zeta(t)] = -w_sC(t) + [C_{in} - C(t)]\frac{dh}{dt} + C(t)\frac{d\zeta}{dt}
\end{equation}

From equation \ref{C_eq2}, we can approximate the solution for concentration numerically.

% In previous model runs, we focused exclusively on $S_M$ as it dominates equation \ref{eq1}. $S_O$, $P$, and $M$ were all set to zero. We will evaluate the need to incorporate these terms and determine the appropriate functions to approximate each.

\subsection*{Model inputs}

We obtained model inputs from field measurements. We observed tidal height, grain size, suspended sediment concentration (SSC), and dry bulk density around Polder 32 over multiple field seasons from 2011 to 2016 \cite{auerbachFloodRiskNatural2015,haleObservationsScalingTidal2019,haleSeasonalVariabilityForces2019}.

For $h$, we extracted one year of contiguous tidal data from from a pressure sensor deployed within the tidal channel near Polder 32. We used the oce package in R (3.6.3) to create an idealized tidal curve from our data \cite{kelleyOceAnalysisOceanographic2020}. The tidal curve was then shifted down so that mean higher high water would be \SI{~0.3}{\meter} above the Sundarban platform and \SI{~1.8}{\meter} above the polder surface ($\zeta$). We replicated this tidal curve for each subsequent year for the length of the model run. Field observations confirm these benchmark elevation \cite{auerbachFloodRiskNatural2015,haleSeasonalVariabilityForces2019,bomerSurfaceElevationSedimentation2020}. In order to simulate sea level rise, the subsequent year tidal curves were increased at a linear rate of \SI{2}{\milli\meter\per\year} which is consistent with field observations.


For $C_{in}$, we use observed values of SSC from Hale et al. \cite{haleObservationsScalingTidal2019} that are characteristic of the tidal channels in the region. Similar to Temmerman et al. \cite{temmermanModellingLongtermTidal2003,temmermanModellingEstuarineVariations2004}, we scaled the observed tidal channel SSC by a factor as the flood waters are expected to have a lower SSC than the tidal channel due to lower flow velocities. For our preliminary study, we use a k-factor of 0.7. In future model iterations, we will better explore this relationship and determine an appropriate k-factor.

For $\rho$, we used values derived from conversations with Steven Goodbred and Carol Wilson.

\matmethods{Please describe your materials and methods here. This can be more than one paragraph, and may contain subsections and equations as required. 

\subsection*{Subsection for Method}
Example text for subsection.
}

\showmatmethods{} % Display the Materials and Methods section

\acknow{Please include your acknowledgments here, set in a single paragraph. Please do not include any acknowledgments in the Supporting Information, or anywhere else in the manuscript.}

\showacknow{} % Display the acknowledgments section

% Bibliography
\bibliography{references}

\end{document}
