\documentclass[9pt,twocolumn,twoside,lineno]{pnas-new}
% Use the lineno option to display guide line numbers if required.
\templatetype{pnasresearcharticle} % Choose template 
% {pnasresearcharticle} = Template for a two-column research article
% {pnasmathematics} %= Template for a one-column mathematics article
% {pnasinvited} %= Template for a PNAS invited submission

\title{Sediment Model Manuscript}

\author[a]{Christopher M. Tasich}
\author[a]{Jonathan Gilligan}
\author[a]{Steven L. Goodbred}

\affil[a]{Vanderbilt University}

% Please give the surname of the lead author for the running footer
\leadauthor{Tasich}

% Please add here a significance statement to explain the relevance of your work
\significancestatement{Authors must submit a 120-word maximum statement about the significance of their research paper written at a level understandable to an undergraduate educated scientist outside their field of speciality. The primary goal of the Significance Statement is to explain the relevance of the work in broad context to a broad readership. The Significance Statement appears in the paper itself and is required for all research papers.}

% Please include corresponding author, author contribution and author declaration information
\authorcontributions{Please provide details of author contributions here.}
\authordeclaration{Please declare any conflict of interest here.}
\correspondingauthor{\textsuperscript{2}To whom correspondence should be addressed. E-mail: chris.tasich\@vanderbilt.edu}

% Keywords are not mandatory, but authors are strongly encouraged to provide them. If provided, please include two to five keywords, separated by the pipe symbol, e.g:
\keywords{Keyword 1 \(|\) Keyword 2 \(|\) Keyword 3 \(|\)} 

\begin{abstract}
	Please provide an abstract of no more than 250 words in a single paragraph. Abstracts should explain to the general reader the major contributions of the article. References in the abstract must be cited in full within the abstract itself and cited in the text.
\end{abstract}

\dates{This manuscript was compiled on \today}
\doi{\url{www.pnas.org/cgi/doi/10.1073/pnas.XXXXXXXXXX}}

\begin{document}
	
	\maketitle	
	\thispagestyle{firststyle}
	\ifthenelse{\boolean{shortarticle}}{\ifthenelse{\boolean{singlecolumn}}{\abscontentformatted}{\abscontent}}{}
	
	% If your first paragraph (i.e. with the \dropcap) contains a list environment (quote, quotation, theorem, definition, enumerate, itemize...), the line after the list may have some extra indentation. If this is the case, add \parshape=0 to the end of the list environment.
	\dropcap{I}ntroduction
	
	\bigskip
	\bigskip
	\bigskip
	\bigskip
	\bigskip
	\bigskip
	\bigskip
	\bigskip
	\bigskip
	\bigskip
	\bigskip
	\bigskip
	\bigskip
	
	
	\section*{Methods}
	
	\subsection*{Field observations and model parameters}
	
	We observed tidal height, grain size, suspended sediment concentration (SSC), and dry bulk density over multiple field seasons (both dry and monsoon) from 2011-2016. The tidal height, measured above the platform, was inferred from a pressure sensor deployed at Polder 32 [INCLUDE LOC]. The median grain size was measured in multiple locations around Polder 32 and within the natural mangrove forest. Grain sizes ranged from 14-27 µm, which is consistent with medium to coarse silt. SSCs were obtained using an optical backscatter point sensor (OBS) affixed to the side of small boat. SSC varied within a tidal cycle (0-3 g/L) and seasonally (0.15-0.77 g/L). Dry bulk density (900-1500 kg/m\textsuperscript{3}) was determined from sediment samples at depths of 50-100 cm \cite{auerbachFloodRiskNatural2015}. These field observations were used to define distributions for each model parameter 
	
	
	\subsection*{Numerical model}
		
	We modeled tidal platform elevation (\(\zeta\)) at a point through time as 	
	\begin{equation}\label{eq1}
	\frac{d \zeta}{d t} = \frac{d S_M}{d t} + \frac{d S_O}{d t} + \frac{d P}{d t} + \frac{d M}{d t},
	\end{equation}
	where (\S_M\) is mineral sedimentation, (\S_O\) is organic matter sedimentation, $P$ is compaction, and $M$ is tectonic subsidence \cite{frenchNumericalSimulationVertical1993,temmermanModellingLongtermTidal2003,temmermanModellingEstuarineVariations2004}.
	
	We approximate $S_M$ as
	\begin{equation}\label{eq2}
	S_M(t) = \int{\frac{w_{s}C(t)}{\rho}dt},
	\end{equation}
	where $w_s$ is the characteristic settling velocity of a grain given by Stokes' law, $C$ is the depth-averaged suspended sediment concentration in the water column, and $\rho$ is the dry bulk density of the sediment. Equation \ref{eq2} assumes there is no resuspension of mineral sediment \cite{kroneMethodSimulatingMarsh1987}.
	
	We capture the temporal variation of suspended sediment concentrations during one tidal cycle through the mass balance given as
	\begin{equation}\label{eq3}
	\frac{d[h(t)-\zeta]C(t)}{dt} = -w_sC(t)+C_{in}\frac{dh}{dt},
	\end{equation}
	where $h$ is the height of the water column and $C_{in}$ is the incoming suspended sediment concentration of the adjacent water column \cite{kroneMethodSimulatingMarsh1987,frenchNumericalSimulationVertical1993,temmermanModellingLongtermTidal2003,temmermanModellingEstuarineVariations2004}.
	
	The incoming suspended sediment concentration also varies with time and can be written as
	\begin{equation}\label{eq4}
	C_{in}(t) = C_{max}[h(t)-\zeta],
	\end{equation}
	where \(C_{max}\) is the maximum suspended sediment concentration of the adjacent water column. 
	
	\subsection*{Monte Carlo simulations}
	
	
	
	\section*{Results}
	
	\section*{Discussion}
	
	\section*{Conclusion}
	
	\section*{Acknowledgments}
	
	\section*{References}
	
	\bibliography{sedmod}
	
\end{document}