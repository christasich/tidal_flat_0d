%% 
%% Copyright 2019-2020 Elsevier Ltd
%% 
%% This file is part of the 'CAS Bundle'.
%% --------------------------------------
%% 
%% It may be distributed under the conditions of the LaTeX Project Public
%% License, either version 1.2 of this license or (at your option) any
%% later version.  The latest version of this license is in
%%    http://www.latex-project.org/lppl.txt
%% and version 1.2 or later is part of all distributions of LaTeX
%% version 1999/12/01 or later.
%% 
%% The list of all files belonging to the 'CAS Bundle' is
%% given in the file `manifest.txt'.
%% 
%% Template article for cas-sc documentclass for 
%% single column output.

%\documentclass[a4paper,fleqn,longmktitle]{cas-sc}
\documentclass[a4paper,fleqn]{cas-sc}

\usepackage{siunitx}

% Document parameters
\DeclareSIUnit\year{yr}
\DeclareSIUnit\foot{ft}

%\usepackage[numbers]{natbib}
%\usepackage[authoryear]{natbib}
\usepackage[authoryear,longnamesfirst]{natbib}

%%%Author macros
\def\tsc#1{\csdef{#1}{\textsc{\lowercase{#1}}\xspace}}
\tsc{WGM}
\tsc{QE}
\tsc{EP}
\tsc{PMS}
\tsc{BEC}
\tsc{DE}
%%%

\begin{document}
\let\WriteBookmarks\relax
\def\floatpagepagefraction{1}
\def\textpagefraction{.001}
\shorttitle{Sediment model}
\shortauthors{C. Tasich et~al.}
%\begin{frontmatter}

\title [mode = title]{This is a specimen $a_b$ title}
\tnotemark[1,2]

\tnotetext[1]{This document is the results of the research
   project funded by the National Science Foundation.}

\tnotetext[2]{The second title footnote which is a longer text matter
   to fill through the whole text width and overflow into
   another line in the footnotes area of the first page.}



\author[1,3]{C. M. Tasich}[type=editor,
                        auid=000,bioid=1,
                        prefix=1,
                        role=Researcher,
                        orcid=0000-0001-7511-2910]
\cormark[1]
\fnmark[1]
\ead{chris.tasich@vanderbilt.edu}
\ead[url]{www.cvr.cc, cvr@sayahna.org}

\credit{Conceptualization of this study, Methodology, Software}

\address[1]{Department of Earth & Environmental Sciences, Vanderbilt University, Nashville TN 37235, USA}

\author[2,4]{Han Theh Thanh}[style=chinese]

\author[2,3]{CV Rajagopal}[%
   role=Co-ordinator,
   suffix=Jr,
   ]
\fnmark[2]
\ead{cvr3@sayahna.org}
\ead[URL]{www.sayahna.org}

\credit{Data curation, Writing - Original draft preparation}

\address[2]{Sayahna Foundation, Jagathy, Trivandrum 695014, India}

\author%
[1,3]
{Rishi T.}
\cormark[2]
\fnmark[1,3]
\ead{rishi@stmdocs.in}
\ead[URL]{www.stmdocs.in}

\address[3]{STM Document Engineering Pvt Ltd., Mepukada,
    Malayinkil, Trivandrum 695571, India}

\cortext[cor1]{Corresponding author}
\cortext[cor2]{Principal corresponding author}
\fntext[fn1]{This is the first author footnote. but is common to third
  author as well.}
\fntext[fn2]{Another author footnote, this is a very long footnote and
  it should be a really long footnote. But this footnote is not yet
  sufficiently long enough to make two lines of footnote text.}

\nonumnote{This note has no numbers. In this work we demonstrate $a_b$
  the formation Y\_1 of a new type of polariton on the interface
  between a cuprous oxide slab and a polystyrene micro-sphere placed
  on the slab.
  }

\begin{abstract}
This template helps you to create a properly formatted \LaTeX\ manuscript.

\noindent\texttt{\textbackslash begin{abstract}} \dots 
\texttt{\textbackslash end{abstract}} and
\verb+\begin{keyword}+ \verb+...+ \verb+\end{keyword}+ 
which
contain the abstract and keywords respectively. 
Each keyword shall be separated by a \verb+\sep+ command.
\end{abstract}

\begin{highlights}
\item Research highlights item 1
\item Research highlights item 2
\item Research highlights item 3
\end{highlights}

\begin{keywords}
quadrupole exciton \sep polariton \sep \WGM \sep \BEC
\end{keywords}

\maketitle

\section{Introduction}

\section{Methods}

We modeled tidal platform elevation ($\zeta$) using a zero-dimensional mass balance model using the basic formulation provided by Krone \citet{kroneMethodSimulatingMarsh1987} and further refined by Allen \citet{allenSaltmarshGrowthStratification1990}, French \citet{frenchNumericalSimulationVertical1993}, and Temmerman et al. \citet{temmermanModellingLongtermTidal2003,temmermanModellingEstuarineVariations2004}. The rate of tidal platform elevation change is described as
\begin{equation}\label{mass_bal_eq}
	\frac{d \eta}{d t} = \frac{d S_M}{d t} + \frac{d S_O}{d t} + \frac{d P}{d t} + \frac{d M}{d t},
\end{equation}
where $S_M$ is mineral sedimentation, $S_O$ is organic matter sedimentation, $P$ is compaction, and $M$ is tectonic subsidence. Each term of equation \ref{mass_bal_eq} can be further expanded.

We approximate $S_M$ as
\begin{equation}\label{min_sed_eq}
	S_M(t) = \int{\frac{w_{s}C(t)}{\rho_b}dt},
\end{equation}
where $w_s$ is the characteristic settling velocity of a given grain size, $C(t)$ is the depth-averaged and time-varying sediment concentration in the water column, and $\rho_b$ is the dry bulk density of the sediment. We assume there is no resuspension of mineral sediment which is consistent with Krone's \citet{kroneMethodSimulatingMarsh1987} initial formulation.

$w_s$ is calculated for a given grain size by using Stokes' law to determine the terminal velocity of a sphere falling through a fluid given by
\begin{equation}\label{ws_eq}
	w_s =  \frac{2}{9}\frac{\rho_p - \rho_f}{\mu}gR^2
\end{equation}

where $\rho_p$ is the mass density of the particle or grain, $\rho_f$ is the mass density of the fluid, $\mu$ is the dynamic viscosity of the fluid, $g$ is the acceleration due to gravity, and $R$ is the radius of the grain. This is approximation for $w_s$ is consistent with previous similar studies \citet{allenSaltmarshGrowthStratification1990,temmermanModellingLongtermTidal2003,temmermanModellingEstuarineVariations2004}. We assume basic properties of water for $\rho_f$ (\SI{1000}{\kilo\gram\per\cubic\meter}) and $\mu$ (\SI{1e-3}{\kilo\gram\per\meter\per\second}) for simplicity. Salinity does vary seasonally which will change these values, but had little affect on the model output.

We capture the temporal variation of $C(t)$ through the mass balance given as
\begin{equation}\label{C_eq}
	\frac{d[h(t)-\zeta(t)]C(t)}{dt} = -w_sC(t)+C_{in}\frac{dh}{dt},
\end{equation}
where $h$ is the height of the water column and $C_{in}$ is the incoming suspended sediment concentration of the adjacent water column. We consider $\zeta$ to be a function of time and update it at every timestep which differs from previous studies \citet{kroneMethodSimulatingMarsh1987,allenSaltmarshGrowthStratification1990,frenchNumericalSimulationVertical1993,temmermanModellingLongtermTidal2003,temmermanModellingEstuarineVariations2004,frenchTidalMarshSedimentation2006} which only update $\zeta$ after every tidal cycle. The physical interpretation of equation \ref{C_eq} is that the first term is the mass flux above an area on the tidal platform, the second term is the mass flux extracted from the water column, and the third term is the mass flux from the adjacent water column. Further derivation of equation \ref{C_eq} results in

\begin{equation}\label{C_eq2}
	\frac{dC}{dt}[h(t) - \zeta(t)] = -w_sC(t) + [C_{in} - C(t)]\frac{dh}{dt} + C(t)\frac{d\zeta}{dt}
\end{equation}

From equation \ref{C_eq2}, we can approximate the solution for concentration numerically.

% In previous model runs, we focused exclusively on $S_M$ as it dominates equation \ref{eq1}. $S_O$, $P$, and $M$ were all set to zero. We will evaluate the need to incorporate these terms and determine the appropriate functions to approximate each.

\subsection*{Model inputs}

We obtained model inputs from field measurements. We observed tidal height, grain size, suspended sediment concentration (SSC), and dry bulk density around Polder 32 over multiple field seasons from 2011 to 2016 \citet{auerbachFloodRiskNatural2015,haleObservationsScalingTidal2019,haleSeasonalVariabilityForces2019}.

For $h$, we extracted one year of contiguous tidal data from from a pressure sensor deployed within the tidal channel near Polder 32. We used the oce package in R (3.6.3) to create an idealized tidal curve from our data \citet{kelleyOceAnalysisOceanographic2020}. The tidal curve was then shifted down so that mean higher high water would be \SI{~0.3}{\meter} above the Sundarban platform and \SI{~1.8}{\meter} above the polder surface ($\zeta$). We replicated this tidal curve for each subsequent year for the length of the model run. Field observations confirm these benchmark elevation \citet{auerbachFloodRiskNatural2015,haleSeasonalVariabilityForces2019,bomerSurfaceElevationSedimentation2020}. In order to simulate sea level rise, the subsequent year tidal curves were increased at a linear rate of \SI{2}{\milli\meter\per\year} which is consistent with field observations.


For $C_{in}$, we use observed values of SSC from Hale et al. \citet{haleObservationsScalingTidal2019} that are characteristic of the tidal channels in the region. Similar to Temmerman et al. \citet{temmermanModellingLongtermTidal2003,temmermanModellingEstuarineVariations2004}, we scaled the observed tidal channel SSC by a factor as the flood waters are expected to have a lower SSC than the tidal channel due to lower flow velocities. For our preliminary study, we use a k-factor of 0.7. In future model iterations, we will better explore this relationship and determine an appropriate k-factor.

For $\rho$, we used values derived from conversations with Steven Goodbred and Carol Wilson.

\section{Results}

\section{Discussion}

\section{Conclusions}

\section{Bibliography}


\appendix


\printcredits

%% Loading bibliography style file
% \bibliographystyle{model1-num-names}
\bibliographystyle{cas-model2-names}
% \bibliographystyle{elsarticle-harv}

% Loading bibliography database
\bibliography{../../PNAS/references}

\end{document}
