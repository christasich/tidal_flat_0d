%% Template Preamble

\documentclass[a4paper,fleqn]{cas-dc}

\usepackage[authoryear,longnamesfirst]{natbib}

%%%Author macros
\def\tsc#1{\csdef{#1}{\textsc{\lowercase{#1}}\xspace}}

%%%

% User-defined preamble
\usepackage{siunitx}

% Document parameters
\DeclareSIUnit\year{yr}
\DeclareSIUnit\foot{ft}

\begin{document}
\let\WriteBookmarks\relax
\def\floatpagepagefraction{1}
\def\textpagefraction{.001}
\shorttitle{Sediment model}
\shortauthors{C.M. Tasich et~al.}

\title [mode = title]{DRAFT: Sediment model}

\author{Christopher M. Tasich}[%
orcid=0000-0001-7511-2910,
suffix=]
\ead{chris.tasich@vanderbilt.edu}
\cormark[1]

\address{Department of Earth \& Environmental Sciences, Vanderbilt University, Nashville TN 37235, USA}

\author{Jonathan M. Gilligan}
\cormark[2]

\author{Steven L. Goodbred}[%
suffix=Jr
]

\begin{abstract}
  The low-lying, coastal region of the Ganges-Brahmaputra (GB) delta has relied on poldering (the creation of embanked islands) to mitigate the effects of tidal inundation and storm surge since the 1960s. The result has been an increase in total habitable and arable land allowing for the sustenance of 20 million people within the tidal deltaplain. However, poldering produced the unintended consequence of starving the interior landscapes of sediment resulting in a significant elevation offset (~1-1.5 m) from that of the natural system. Engineering efforts, such as tidal river management (TRM), propose a controlled inundation effort to allow sediment exchange with the tidal network. Some local TRM efforts have succeeded, while other have not. However, there have been few quantitative analyses aimed at understanding the relationship between tidal inundation and sediment accumulation.  Furthermore, sea level rise (SLR) and decreases in suspended sediment concentrations (SSC) due to damming of rivers may also affect sediment accumulations in the future. We use a combination of field based observations and modeling to simulate the long-term evolution of both the poldered and the natural system in the GB delta.

  Our model employs a mass balance with sediment accumulation controlled by tidal height above the platform, SSC, settling velocity, and dry bulk density. Tidal height is determine using pressure sensor data with projected SLR superimposed. SSC varies within both one tidal cycle (0-3 g/L) and seasonally (0.15-0.77 g/L). Grain size (14-27 µm) is used as a proxy for determining settling velocity. Dry bulk density (900-1500 kg/m3) is determined from sediment samples at depths of 50-100 cm. We use a Monte Carlo simulation to project sediment accumulation probabilities over the next century. Furthermore, we simulate perturbations to the system such as decreases in SSC due to recent damming of the Ganges in India. Baseline results suggest the P32 system could recover to that of the natural system in only 7 years. However, aggressive SLR projections or decreases in SSC result in mean high water out-pacing sediment accumulation for both P32 and the natural mangrove forest.
\end{abstract}

% \begin{highlights}
% \item SSC is sufficient to maintain pace with SLR.
% \end{highlights}

\begin{keywords}
TRM \sep polders \sep sea level rise
\end{keywords}

\maketitle

\section{Introduction}

\section{Methods}

We modeled tidal platform elevation ($\zeta$) using a zero-dimensional mass balance model using the basic formulation provided by Krone \citet{kroneMethodSimulatingMarsh1987} and further refined by Allen \citet{allenSaltmarshGrowthStratification1990}, French \citet{frenchNumericalSimulationVertical1993}, and Temmerman et al. \citet{temmermanModellingLongtermTidal2003,temmermanModellingEstuarineVariations2004}. The rate of tidal platform elevation change is described as
\begin{equation}\label{mass_bal_eq}
	\frac{d \eta}{d t} = \frac{d S_M}{d t} + \frac{d S_O}{d t} + \frac{d P}{d t} + \frac{d M}{d t},
\end{equation}
where $S_M$ is mineral sedimentation, $S_O$ is organic matter sedimentation, $P$ is compaction, and $M$ is tectonic subsidence. Each term of equation \ref{mass_bal_eq} can be further expanded.

We approximate $S_M$ as
\begin{equation}\label{min_sed_eq}
	S_M(t) = \int{\frac{w_{s}C(t)}{\rho_b}dt},
\end{equation}
where $w_s$ is the characteristic settling velocity of a given grain size, $C(t)$ is the depth-averaged and time-varying sediment concentration in the water column, and $\rho_b$ is the dry bulk density of the sediment. We assume there is no resuspension of mineral sediment which is consistent with Krone's \citet{kroneMethodSimulatingMarsh1987} initial formulation.

$w_s$ is calculated for a given grain size by using Stokes' law to determine the terminal velocity of a sphere falling through a fluid given by
\begin{equation}\label{ws_eq}
	w_s =  \frac{2}{9}\frac{\rho_p - \rho_f}{\mu}gR^2
\end{equation}

where $\rho_p$ is the mass density of the particle or grain, $\rho_f$ is the mass density of the fluid, $\mu$ is the dynamic viscosity of the fluid, $g$ is the acceleration due to gravity, and $R$ is the radius of the grain. This is approximation for $w_s$ is consistent with previous similar studies \citet{allenSaltmarshGrowthStratification1990,temmermanModellingLongtermTidal2003,temmermanModellingEstuarineVariations2004}. We assume basic properties of water for $\rho_f$ (\SI{1000}{\kilo\gram\per\cubic\meter}) and $\mu$ (\SI{1e-3}{\kilo\gram\per\meter\per\second}) for simplicity. Salinity does vary seasonally which will change these values, but had little affect on the model output.

We capture the temporal variation of $C(t)$ through the mass balance given as
\begin{equation}\label{C_eq}
	\frac{d[h(t)-\zeta(t)]C(t)}{dt} = -w_sC(t)+C_{in}\frac{dh}{dt},
\end{equation}
where $h$ is the height of the water column and $C_{in}$ is the incoming suspended sediment concentration of the adjacent water column. We consider $\zeta$ to be a function of time and update it at every timestep which differs from previous studies \citet{kroneMethodSimulatingMarsh1987,allenSaltmarshGrowthStratification1990,frenchNumericalSimulationVertical1993,temmermanModellingLongtermTidal2003,temmermanModellingEstuarineVariations2004,frenchTidalMarshSedimentation2006} which only update $\zeta$ after every tidal cycle. The physical interpretation of equation \ref{C_eq} is that the first term is the mass flux above an area on the tidal platform, the second term is the mass flux extracted from the water column, and the third term is the mass flux from the adjacent water column. Further derivation of equation \ref{C_eq} results in

\begin{equation}\label{C_eq2}
	\frac{dC}{dt}[h(t) - \zeta(t)] = -w_sC(t) + [C_{in} - C(t)]\frac{dh}{dt} + C(t)\frac{d\zeta}{dt}
\end{equation}

From equation \ref{C_eq2}, we can approximate the solution for concentration numerically.

% In previous model runs, we focused exclusively on $S_M$ as it dominates equation \ref{eq1}. $S_O$, $P$, and $M$ were all set to zero. We will evaluate the need to incorporate these terms and determine the appropriate functions to approximate each.

\subsection*{Model inputs}

We obtained model inputs from field measurements. We observed tidal height, grain size, suspended sediment concentration (SSC), and dry bulk density around Polder 32 over multiple field seasons from 2011 to 2016 \citet{auerbachFloodRiskNatural2015,haleObservationsScalingTidal2019,haleSeasonalVariabilityForces2019}.

For $h$, we extracted one year of contiguous tidal data from from a pressure sensor deployed within the tidal channel near Polder 32. We used the oce package in R (3.6.3) to create an idealized tidal curve from our data \citet{kelleyOceAnalysisOceanographic2020}. The tidal curve was then shifted down so that mean higher high water would be \SI{~0.3}{\meter} above the Sundarban platform and \SI{~1.8}{\meter} above the polder surface ($\zeta$). We replicated this tidal curve for each subsequent year for the length of the model run. Field observations confirm these benchmark elevation \citet{auerbachFloodRiskNatural2015,haleSeasonalVariabilityForces2019,bomerSurfaceElevationSedimentation2020}. In order to simulate sea level rise, the subsequent year tidal curves were increased at a linear rate of \SI{2}{\milli\meter\per\year} which is consistent with field observations.


For $C_{in}$, we use observed values of SSC from Hale et al. \citet{haleObservationsScalingTidal2019} that are characteristic of the tidal channels in the region. Similar to Temmerman et al. \citet{temmermanModellingLongtermTidal2003,temmermanModellingEstuarineVariations2004}, we scaled the observed tidal channel SSC by a factor as the flood waters are expected to have a lower SSC than the tidal channel due to lower flow velocities. For our preliminary study, we use a k-factor of 0.7. In future model iterations, we will better explore this relationship and determine an appropriate k-factor.

For $\rho$, we used values derived from conversations with Steven Goodbred and Carol Wilson.

\section{Results}

\section{Discussion}

\section{Conclusions}

\section{Bibliography}


\appendix


\printcredits

%% Loading bibliography style file
% \bibliographystyle{model1-num-names}
\bibliographystyle{cas-model2-names}
% \bibliographystyle{elsarticle-harv}

% Loading bibliography database
\bibliography{references}

\end{document}
