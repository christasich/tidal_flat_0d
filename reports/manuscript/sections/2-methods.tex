\section{Methods}

\subsection{Model design}

We modeled the vertical accretion of a tidal platform ($\sfrac{d\eta}{dt}$) using a zero-dimensional mass balance approach initially described by \citet{kroneMethodSimulatingMarsh1987} and validated by subsequent studies \citep{allenSaltmarshGrowthStratification1990, frenchNumericalSimulationVertical1993, temmermanModellingLongtermTidal2003,temmermanModellingEstuarineVariations2004}. The rate of vertical accretion is described as
\begin{equation}\label{eq:mass_bal}
	\frac{d\eta(t)}{dt} = \frac{dS_m(t)}{dt} + \frac{dS_o(t)}{dt} + \frac{dP(t)}{dt} + \frac{dM(t)}{dt},
\end{equation}
where $\sfrac{dS_m(t)}{dt}$ is the rate of mineral sedimentation, $\sfrac{dS_o(t)}{dt}$ is the rate of organic matter sedimentation, $\sfrac{dP(t)}{dt}$ is the rate of compaction of the deposited sediment, and $\sfrac{dM(t)}{dt}$ is the rate of tectonic subsidence. Studies \todo{Need ref. Could also use a proportion of total sediment accumulation. Maybe unncessary. It likely gets internalized after model calibration. Maybe just combine $S_m$ and $S_o$?} have shown only \SI{\sim3}{\percent} of bulk sedimentation in the region comes from organic matter and therefore we chose to set $\sfrac{dS_o(t)}{dt}$ to zero. We also effectively internalized compaction by using dry bulk density within the mineral sedimentation term so we also set $\sfrac{dP(t)}{dt}$ to zero. Lastly, we set the $\sfrac{dM(t)}{dt}$ to \SI{6}{\milli\meter} which is consistent with measured rates of tectonic subsidence for the region \cite{higginsInSARMeasurementsCompaction2014a} \todo{Need to investigate further. Look into Steckler papers from 2010 and 2013.}. The remaining mineral sedimentation terms varies within a tidal cycle and requires additional steps.

In order to solve $\sfrac{dS_m(t)}{dt}$, we began by conceptualizing a tidal platform periodically inundated by cyclical tides. We first defined depth to be
\begin{equation}\label{eq:depth}
	h(t) = \zeta(t) - \eta(t),
\end{equation}
where $\zeta(t)$ is the water-surface elevation and $\eta(t)$ is the sediment-surface elevation which also implies that
\begin{equation}\label{eq:depth_dt}
	\frac{dh(t)}{dt} = \frac{d\zeta(t)}{dt} - \frac{d\eta(t)}{dt}.
\end{equation}
Independently, we assume when $h(t) > 0$, the rate of mineral sedimentation is
\begin{equation}\label{eq:sed_flux}
	\frac{dS_m(t)}{dt} = \frac{w_sC(t)}{\rho_b},
\end{equation}
where $w_s$ is the nominal settling velocity of a sediment grain, $C(t)$ is the depth-averaged suspended sediment concentration (SSC) in the water column, and $\rho_b$ is the bulk density of the sediment. We assumed no resuspension of mineral sediment which is practical and consistent with previous studies \citep{kroneMethodSimulatingMarsh1987, allenSaltmarshGrowthStratification1990, frenchNumericalSimulationVertical1993, temmermanModellingLongtermTidal2003, temmermanModellingEstuarineVariations2004}. We used Stoke's law to determine $w_s$. Stoke's law assumes unhindered settling which likely overestimates actual settling rates and, therefore, mineral sedimentation rates. However, we only considered settling for a singular, median grain size which likely underestimated mineral sedimentation rates from coarser grains. Model calibration further corrected for these errors. Thus, the $w_s$ given by Stoke's law should be considered an imprecise, but reasonable approximation. \todo{Need ref.}

In order to solve for $C(t)$ in \cref{eq:sed_flux}, we first defined a mass balance of sediment within the water column as
\begin{equation}\label{eq:conc_mass_bal}
	\frac{d}{dt}[h(t)C(t)] = -w_s C(t) + C_b \frac{dh(t)}{dt},
\end{equation}
which can be expanded and rerranged as
\begin{equation}\label{eq:conc_dt}
	\frac{dC(t)}{dt} = - \frac{w_sC(t)}{h(t)} - \frac{1}{h(t)}[C(t) - C_b]\frac{dh(t)}{dt}.
\end{equation}
The mass flux from the boundary term was constrained to only occur during flood tide ($\sfrac{d\zeta}{dt} > 0$) which is consistent with previous studies \citep{kroneMethodSimulatingMarsh1987, allenSaltmarshGrowthStratification1990, frenchNumericalSimulationVertical1993, temmermanModellingLongtermTidal2003, temmermanModellingEstuarineVariations2004}. We formalized this mathematically using a Heaviside function which serves as a binary switch controlling the flux of sediment from the boundary. The Heaviside function is given as
\begin{equation}\label{eq:heaviside}
	S = \frac{d\zeta}{dt},\\
	H(S) =
	\begin{cases}
		0 & \text{if $S < 0$}\\
		1 & \text{if $S \geq 0$}.
	\end{cases}
\end{equation}
\Cref{eq:conc_dt} then becomes
\begin{equation}\label{eq:conc_sol}
	\frac{dC(t)}{dt} = - \frac{w_s C(t)}{h(t)} - \frac{H(S)}{h(t)} [C(t) - C_b]\frac{dh(t)}{dt}.
\end{equation}

Finally, we solved \crefnosort{eq:conc_sol,eq:sed_flux,eq:mass_bal} in that order. Our apporoach differs from previous studies in that we resolve each equation at every timestep. We do this by using a computationally efficient Runge-Kutta method which allows for a variable time step. The system of equations was solved using an explicit Runge-Kutta method of order 5(4) \citep{dormandFamilyEmbeddedRungeKutta1980} and implemented in Python using SciPy \citep{virtanenSciPyFundamentalAlgorithms2020}.

% \subsection{Model inputs}

% We obtained model inputs from field measurements. We observed tidal height, grain size, suspended sediment concentration (SSC), and dry bulk density around Polder 32 over multiple field seasons from 2011 to 2016 \citep{auerbachFloodRiskNatural2015,haleObservationsScalingTidal2019,haleSeasonalVariabilityForces2019}.

% For $h$, we extracted one year of contiguous tidal data from from a pressure sensor deployed within the tidal channel near Polder 32. We used the oce package in R (3.6.3) to create an idealized tidal curve from our data \citep{kelleyOceAnalysisOceanographic2020}. The tidal curve was then shifted down so that mean higher high water would be \SI{~0.3}{\meter} above the Sundarban platform and \SI{~1.8}{\meter} above the polder surface ($\zeta$). We replicated this tidal curve for each subsequent year for the length of the model run. Field observations confirm these benchmark elevation \citep{auerbachFloodRiskNatural2015,haleSeasonalVariabilityForces2019,bomerSurfaceElevationSedimentation2020}. In order to simulate sea level rise, the subsequent year tidal curves were increased at a linear rate of \SI{2}{\milli\meter\per\year} which is consistent with field observations.


% For $C_{in}$, we use observed values of SSC from Hale et al. \citep{haleObservationsScalingTidal2019} that are characteristic of the tidal channels in the region. Similar to Temmerman et al. \citep{temmermanModellingLongtermTidal2003,temmermanModellingEstuarineVariations2004}, we scaled the observed tidal channel SSC by a factor as the flood waters are expected to have a lower SSC than the tidal channel due to lower flow velocities. For our preliminary study, we use a k-factor of 0.7. In future model iterations, we will better explore this relationship and determine an appropriate k-factor.

% For $\rho$, we used values derived from conversations with Steven Goodbred and Carol Wilson.