\section{Introduction}

Sea level rise threatens the densely populated and ecologically significant low-lying, coastal region of Bengal. Mitigating the effects of sea level rise (SLR) in this region will be especially difficult considering current widespread land management practices and the ongoing geopolitical tension between India and Bangladesh.

Coastal Bengal is situated within the delta formed by the confluence of the Ganges and Brahmaputra rivers and straddles the border between West Bengal, India to the east and Bangladesh to the west. The region is home to \textasciitilde30 million people \citep{centerforinternationalearthscienceinformationnetwork-ciesin-columbiauniversityPopulationEstimationService2018} and the ecologically critical Sundarbans mangrove forest. Since the 1960s, the region has seen a vast transformation through the reduction in natural mangrove habitat and the widspread construction of earthen embankments. These embankments surround large swaths of clearcut land, known as polders, that accomodate the swelling population of the region.

While, preventing regular inundation by spring high tides, the creation of these embankments have had the unintended consequence of starving the interior of the polders of fresh sediment. Without this sediment, polder interiors have compacted resulting in signficant elevation offset (\SIrange{1.0}{1.5}{\meter}) relative to the natural mangrove forest \citep{auerbachFloodRiskNatural2015}. Polder elevations often sit precariously below local mean high water (MHW) levels leading to persistent waterlogging. Furthermore, many of these embankments are in disrepair and susceptible to breaching especially by storm surge as was the case with Cyclone Sidr (2007), Cyclone Aila (2009), and recently with Cyclone Amphan (2020).

Tidal river management (TRM) has been proposed as a possible augmentation of current land management practices to alleviate some of the issues caused by poldering. Under TRM, embankments are seasonally breached to allow tidal water inundation and sediment aggradation. Many low-lying areas may recover to an acceptable elevation within 5 to 10 years, though, some may take longer. There are a host of local socioeconomic, political, and governance consideration that may influence the success of TRM. Here, we neglect those considerations and focus on the general feasibility of TRM in regards to SLR and the sediment supply of the GB system. Future studies will focus on the social dynamics surround TRM.

Coastal Bengal is often seen as one of the most at-risk regions for SLR due to climate change. Infographics often depict large swaths of the Bengal coastline flooded under different SLR scenarios. However, this overly simplifies the threats to the region and neglects the significant sediment contribution of the GB system in mainintaining the natural elevation.

Estimates for increases in Relative Mean Sea Level (RMSL) in the GB delta range from \SIrange{2.8}{8.8}{\milli\meter\per\year}. However, RMSL neglects the widening of the tidal range in the polder region. On average, local high water levels in the polder region are increasing at a rate of \SI{15.9}{\milli\meter\per\year} \citep{pethickRapidRiseEffective2013}. While SLR is of paramount importance, tidal range amplification is the more imminent threat to the region. This is especially important considering the Bangladeshi government's recent reinvestment in poldering with a \$400 million loan from the World Bank for the Coastal Embankment Improvement Project - Phase I (CEIP-I).

As for the natural mangrove system, it is unclear how changing water levels will affect elevation. Some studies have shown that the region is incredibly resilient to increasing water levels due, in large part, to the abudant sediment supply of the GB system. This sediment is delivered to the platform periodically during spring high tide which helps maintain an equilibirum elevation approximately equivalent to mean higher high water (MHHW). But, this large volume of sediment delivered to coastal Bengal is not guaranteed.

Water has long been the focus of the geopolitical disputes between India and Bangladesh. However, the reduction in waterflow across the border portends a significant decrease in sediment flux. Estimates suggest sediment flux may be reduced by \SIrange{39}{75}{\percent} for the Ganges and \SIrange{9}{25}{\percent} for the Brahmaputra resulting in a change in aggradation from \SI{3.6}{\milli\meter\per\year} to \SI{2.5}{\milli\meter\per\year} \citep{higginsRiverLinkingIndia2018}.

The combination of increasing water levels and decreasing sediment supply may further intensify an already dire situation. Here, we use a zero-dimensional mass balance model of sediment aggradation to understand the impact that increasing water levels and decreasing sediment flux will have on the regions equilibrium elevation and consequently its resilience to climate change. We consider both the resilience of the natural mangrove system and the ability of the polder system to recover to a more resilient elevation through TRM.