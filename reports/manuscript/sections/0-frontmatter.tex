\title[mode = title]{DRAFT: Sediment model}

\author{Christopher M. Tasich}[
	orcid=0000-0001-7511-2910,
  twitter=ctasich,
  linkedin=christasich,
	suffix=]
\ead{chris.tasich@vanderbilt.edu}
\cormark[1]

\author{Jonathan M. Gilligan}
\ead{jonathan.gilligan@vanderbilt.edu}
\cormark[2]

\author{Steven L. Goodbred}[
  orcid=0000-0001-7511-2910,
	suffix=Jr]
\ead{steven.goodbred@vanderbilt.edu}

\address{Department of Earth \& Environmental Sciences, Vanderbilt University, Nashville TN 37235, USA}

\cortext[cor1]{Corresponding author}
\cortext[cor2]{Principal corresponding author}

\begin{abstract}
    The low-lying, coastal region of the Ganges-Brahmaputra (GB) delta has relied on poldering (the creation of embanked islands) to mitigate the effects of tidal inundation and storm surge since the 1960s. The result has been an increase in total habitable and arable land allowing for the sustenance of 20 million people within the tidal deltaplain. However, poldering produced the unintended consequence of starving the interior landscapes of sediment resulting in a significant elevation offset (~1-1.5 m) from that of the natural system. Engineering efforts, such as tidal river management (TRM), propose a controlled inundation effort to allow sediment exchange with the tidal network. Some local TRM efforts have succeeded, while other have not. However, there have been few quantitative analyses aimed at understanding the relationship between tidal inundation and sediment accumulation.  Furthermore, sea level rise (SLR) and decreases in suspended sediment concentrations (SSC) due to damming of rivers may also affect sediment accumulations in the future. We use a combination of field based observations and modeling to simulate the long-term evolution of both the poldered and the natural system in the GB delta.

    Our model employs a mass balance with sediment accumulation controlled by tidal height above the platform, SSC, settling velocity, and dry bulk density. Tidal height is determine using pressure sensor data with projected SLR superimposed. SSC varies within both one tidal cycle (0-3 g/L) and seasonally (0.15-0.77 g/L). Grain size (14-27 µm) is used as a proxy for determining settling velocity. Dry bulk density (900-1500 kg/m3) is determined from sediment samples at depths of 50-100 cm. We use a Monte Carlo simulation to project sediment accumulation probabilities over the next century. Furthermore, we simulate perturbations to the system such as decreases in SSC due to recent damming of the Ganges in India. Baseline results suggest the P32 system could recover to that of the natural system in only 7 years. However, aggressive SLR projections or decreases in SSC result in mean high water out-pacing sediment accumulation for both P32 and the natural mangrove forest.
  \end{abstract}

\begin{keywords}
  TRM \sep polders \sep sea level rise
\end{keywords}

\begin{highlights}
  \item Research highlights item 1
  \item Research highlights item 2
  \item Research highlights item 3
\end{highlights}

\maketitle