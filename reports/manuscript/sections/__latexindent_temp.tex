\section{Methods}

\subsection{Model design}

We modeled the vertical accretion of a tidal platform ($\sfrac{d\eta}{dt}$) using a zero-dimensional mass balance approach initially described by \citet{kroneMethodSimulatingMarsh1987} and validated by subsequent studies \citep{allenSaltmarshGrowthStratification1990,frenchNumericalSimulationVertical1993,temmermanModellingLongtermTidal2003,temmermanModellingEstuarineVariations2004}. The rate of vertical accretion is described as
\begin{equation}\label{eq:mass_bal}
	\frac{d\eta(t)}{dt} = \frac{dS_m(t)}{dt} + \frac{dS_o(t)}{dt} + \frac{dP(t)}{dt} + \frac{dM(t)}{dt},
\end{equation}
where $\sfrac{dS_m(t)}{dt}$ is the rate of mineral sedimentation, $\sfrac{dS_o(t)}{dt}$ is the rate of organic matter sedimentation, $\sfrac{dP(t)}{dt}$ is the rate of shallow compaction after dewatering of the deposited sediment, and $\sfrac{dM(t)}{dt}$ is the rate of tectonic subsidence. We considered $\sfrac{dS_o(t)}{dt}$, $\sfrac{dP(t)}{dt}$, $\sfrac{dM(t)}{dt}$ to be constants and used characteristic yearly rates for each; while, $\sfrac{dS_m(t)}{dt}$ varies within a tidal cycle \citep{haleSeasonalVariabilityForces2019} and requires additional treatment.

To solve for $\sfrac{dS_m(t)}{dt}$, we began by conceptualizing a tidal platform periodically inundated by sinusoidal tides. We first defined depth to be
\begin{equation}\label{eq:depth}
	h(t) = \zeta(t) - \eta(t),
\end{equation}
where $\zeta(t)$ is the water-surface elevation and $\eta(t)$ is the sediment-surface elevation which also implies that
\begin{equation}\label{eq:depth_dt}
	\frac{dh(t)}{dt} = \frac{d\zeta(t)}{dt} - \frac{d\eta(t)}{dt}.
\end{equation}
Independently, we assume while $h(t) > 0$, the rate of mineral sedimentation is
\begin{equation}\label{eq:sed_flux}
	\frac{dS_m(t)}{dt} = \frac{w_sC(t)}{\rho_b},
\end{equation}
where $w_s$ is the nominal settling velocity of a sediment grain, $C(t)$ is the depth-averaged suspended sediment concentration (SSC) in the water column, and $\rho_b$ is the bulk density of the sediment. We assumed no resuspension of mineral sediment which is practical and consistent with previous studies \citep{kroneMethodSimulatingMarsh1987, allenSaltmarshGrowthStratification1990, frenchNumericalSimulationVertical1993, temmermanModellingLongtermTidal2003, temmermanModellingEstuarineVariations2004}.

In order to solve for $C(t)$ in \cref{eq:sed_flux}, we first defined a mass balance of sediment within the water column as
\begin{equation}\label{eq:conc_mass_bal}
	\frac{d}{dt}[h(t)C(t)] = -w_s C(t) + C_b \frac{dh(t)}{dt},
\end{equation}
which can be expanded and rerranged as
\begin{equation}\label{eq:conc_dt}
	\frac{dC(t)}{dt} = - \frac{w_sC(t)}{h(t)} - \frac{1}{h(t)}[C(t) - C_b]\frac{dh(t)}{dt}.
\end{equation}
We assumed advection of new sediment to only occur during flood tide by constraining mass flux from the boundary term when $\sfrac{dh}{dt} > 0$. We formalized this mathematically using a Heaviside function which serves as a binary switch and is given as
\begin{equation}\label{eq:heaviside}
	S = \frac{dh}{dt},\quad	H(S) =
	\begin{cases}
		0 & \text{if $S < 0$}\\
		1 & \text{if $S \geq 0$}.
	\end{cases}
\end{equation}
\Cref{eq:conc_dt} then becomes
\begin{equation}\label{eq:conc_sol}
	\frac{dC(t)}{dt} = - \frac{w_s C(t)}{h(t)} - \frac{H(S)}{h(t)} [C(t) - C_b]\frac{dh(t)}{dt}.
\end{equation}
\Crefnosort{eq:conc_sol,eq:sed_flux,eq:mass_bal} were then solved in that order to obtain the change in elevation during one time step.

We integrated this series of equations for each inundation cycle using an explicit Runge-Kutta method of order 5(4) \citep{dormandFamilyEmbeddedRungeKutta1980} implemented in Python using SciPy \citep{virtanenSciPyFundamentalAlgorithms2020}. We used an adaptive step size which provided computational efficiency by decreasing step size as needed - i.e. beginning and end of an inundation cycle. To avoid numerical errors due to very small depths in \cref{eq:conc_sol}, we only allowed the model to integrate while water depths were \SI{>1}{\milli\meter}. Outside of the integration (i.e. while the platform was dry), we continued to apply linear rates for $\sfrac{dS_o(t)}{dt}$, $\sfrac{dP(t)}{dt}$, $\sfrac{dM(t)}{dt}$.

We indentified indundation cycles by filtering the tidal curve for water-surface elevations that were above the corresponding sediment-surface elevation. The time of first element of the filtered data was used to initialize the integration. The adaptive step size method required a continuous function for water-surface elevations so we converted the tidal data to an interpolated univariate spline during the integration. The integration continued until the water-surface elevation fell below the sediment-surface elevation. We repeated this process for all subsequent inundation cycles through the prescribed length of each simulation to obtain a final elevation.

\subsection{Field observations and model parameters}

\bigskip

\subsubsection*{Tidal data}

The tidal curve was derived from observations at Sutarkhali station. Water-surface elevations were collected every 10 minutes from January 1, 2019 to December 31, 2019 using an Onset U20L-01 HOBO water level data logger. The data were processed and upsampled to \SI{1}{\second} temporal resolution using the oce package in R (3.6.3) \citep{kelleyOceAnalysisOceanographic2020}. The tidal curve was then shifted to place mean water at \SI{\approx1.6}{\meter} below the Sundarban sediment-surface. This results in \SIrange{\approx 50}{60}{\centi\meter} of inundation during most spring high tides and is consistent with survey data \citep{auerbachFloodRiskNatural2015,haleSeasonalVariabilityForces2019,bomerSurfaceElevationSedimentation2020}. We repeated this tidal curve for each subsequent year of the simulation with a superimposed sea level rise rate of \SI{5}{\milli\meter\per\year}.

\subsubsection*{Organic matter, shallow compaction, and subsidence}

\citet{bomerProcessControlsLive2020} found organic matter only accounts for $\sim$\SI{0.9(1)}{\percent} of the $\sim$\SI{2.42(26)}{\centi\meter\per\year} yearly bulk sedimentation. Using this, we set $\sfrac{dS_o(t)}{dt}$ to \SI{0.2}{\milli\meter\per\year}.

Shallow compaction is difficult to constrain for the region and likely varies significantly due to natural variability in stratigraphy and anthropogenic activity (e.g. subsurface fluid extraction and accelerated oxidation of below-ground biomass due to drying of the sediment). \citet{auerbachFloodRiskNatural2015} suggested shallow compaction rates of \SI{0.4}{\centi\meter\per\year} for natural compaction and \SI{0.8}{\centi\meter\per\year} for accelerated compaction (combined natural and anthropogenic compaction). We adopted their values of $\sfrac{dP(t)}{dt}$.

Estimates of subsidence vary for the region and are hard to disentangle from compaction. Many studies \citep{pethickRapidRiseEffective2013,goodbredSignificanceLargeSediment2000,stanleyHoloceneDepositionalPatterns2000} combine compaction and subsidence. \citet{auerbachFloodRiskNatural2015} considered both compaction and subsidence separately by using values in the literature \citep{pethickRapidRiseEffective2013,goodbredSignificanceLargeSediment2000,stanleyHoloceneDepositionalPatterns2000} estimated subsidence to be \SI{0.3}{\centi\meter\per\year}. We set $\sfrac{dM(t)}{dt}$ to this value.

\subsubsection*{Settling velocity, suspend sediment concentration, and bulk density}

We used Stoke's law to determine $w_s$. Stoke's law assumes unhindered settling which likely overestimates actual settling rates and, therefore, mineral sedimentation rates. However, we only considered settling for a singular, median grain size which likely underestimated mineral sedimentation rates from coarser grains. Model calibration further corrected for these errors. Thus, the $w_s$ given by Stoke's law should be considered an imprecise, but reasonable approximation.

For $C_{b}$, we use observed values of SSC from \citet{haleObservationsScalingTidal2019} that are characteristic of the tidal channels in the region.

For $\rho$, we used values derived from conversations with Steven Goodbred and Carol Wilson.